%%%%%%%%%%%%%%%%%%%%%%%%%%%%%%%%%%%%%%%%%%%%%%%%%%%%%%%%%%%%%%%%%%%%%%%%%%%%%%%%%%%%%%
% Modelo de relatório de Disciplina de MLP a partir da
% classe latex iiufrgs disponivel em http://github.com/schnorr/iiufrgs
%%%%%%%%%%%%%%%%%%%%%%%%%%%%%%%%%%%%%%%%%%%%%%%%%%%%%%%%%%%%%%%%%%%%%%%%%%%%%%%%%%%%%%

%%%%%%%%%%%%%%%%%%%%%%%%%%%%%%%%%%%%%%%%%%%%%%%%%%%%%%%%%%%%%%%%%%%%%%%%%%%%%%%%%%%%%%
% Definição do tipo / classe de documento e estilo usado
%%%%%%%%%%%%%%%%%%%%%%%%%%%%%%%%%%%%%%%%%%%%%%%%%%%%%%%%%%%%%%%%%%%%%%%%%%%%%%%%%%%%%%
%
\documentclass[rel_mlp]{iiufrgs}

%%%%%%%%%%%%%%%%%%%%%%%%%%%%%%%%%%%%%%%%%%%%%%%%%%%%%%%%%%%%%%%%%%%%%%%%%%%%%%%%%%%%%%
% Importação de pacotes
%%%%%%%%%%%%%%%%%%%%%%%%%%%%%%%%%%%%%%%%%%%%%%%%%%%%%%%%%%%%%%%%%%%%%%%%%%%%%%%%%%%%%%
% (a A seguir podem ser importados os pacotes necessários para o documento, de acordo
% com a necessidade)
%
\usepackage[brazilian]{babel}	    % para texto escrito em pt-br
\usepackage[utf8]{inputenc}         % pacote para acentuação
\usepackage{graphicx}         	    % pacote para importar figuras
\usepackage[T1]{fontenc}            % pacote para conj. de caracteres correto
\usepackage{times}                  % pacote para usar fonte Adobe Times
\usepackage{enumerate}              % para lista de itens com letras
\usepackage{breakcites}
\usepackage{titlesec}
\usepackage{enumitem}
\usepackage{titletoc}
\usepackage{listings}			    % para listagens de código-fonte
\usepackage{mathptmx}               % p/ usar fonte Adobe Times nas formulas matematicas
\usepackage{url}                    % para formatar URLs
%\usepackage{color}				    % para imagens e outras coisas coloridas
%\usepackage{fixltx2e}              % para subscript
%\usepackage{amsmath}               % para \epsilon e matemática
%\usepackage{amsfonts}
%\usepackage{setspace}			    % para mudar espaçamento dos parágrafos
%\usepackage[table,xcdraw]{xcolor}  % para tabelas coloridas
%\usepackage{longtable}             % para tabelas compridas (mais de uma página)
%\usepackage{float}
%\usepackage{booktabs}
%\usepackage{tabularx}
%\usepackage{hyperref}

\usepackage[alf,abnt-emphasize=bf]{abntex2cite}	% pacote para usar citações abnt

% isso meio que coloca helvetica no título dos capítulos, mas fica com muito espaçamento
%\titleformat{\chapter}
%  {\normalfont\fontsize{12}{0}\sffamily\bfseries}
%  {\thechapter}
%  {1em}
%  {}


% isso muda a fonte de tudo:
% \renewcommand\familydefault{\sfdefault}
\newcommand\tab[1][1cm]{\hspace*{#1}}

%%%%%%%%%%%%%%%%%%%%%%%%%%%%%%%%%%%%%%%%%%%%%%%%%%%%%%%%%%%%%%%%%%%%%%%%%%%%%%%%%%%%%%
% Macros, ajustes e definições
%%%%%%%%%%%%%%%%%%%%%%%%%%%%%%%%%%%%%%%%%%%%%%%%%%%%%%%%%%%%%%%%%%%%%%%%%%%%%%%%%%%%%%
%

% define estilo de parágrafo para citação longa direta:
\newenvironment{citacao}{
    %\singlespacing
    %\footnotesize
    \small
    \begin{list}{}{
        \setlength{\leftmargin}{4.0cm}
        \setstretch{1}
        \setlength{\topsep}{1.2cm}
        \setlength{\listparindent}{\parindent}
    }
    \item[]}{\end{list}
}

% adiciona a fonte em figuras e tabelas
\newcommand{\fonte}[1]{\\Fonte: {#1}}

% Ative o seguinte caso alguma nota de rodapé fique muito longa e quebre entre múltiplas
% páginas
%\interfootnotelinepenalty=10000

%%%%%%%%%%%%%%%%%%%%%%%%%%%%%%%%%%%%%%%%%%%%%%%%%%%%%%%%%%%%%%%%%%%%%%%%%%%%%%%%%%%%%%
% Informações gerais
%%%%%%%%%%%%%%%%%%%%%%%%%%%%%%%%%%%%%%%%%%%%%%%%%%%%%%%%%%%%%%%%%%%%%%%%%%%%%%%%%%%%%%

% título
\title{Implementação de um Tower Defense em Swift - Entrega parcial}

% autor
%\author{Autor2}{Aluno} % {sobrenome}{nome} 1 para cada aluno
\author{Boranga}{Augusto} % {sobrenome}{nome}
\author{Franzoi Scroferneker}{Rodrigo} % {sobrenome}{nome}

% Professor orientador da disciplina
\advisor[Prof.~Dr.]{Mello Schnorr}{Lucas}

% Nome do(s) curso(s):
\course{Curso de Graduação em Ciência da Computa{\c{c}}{\~a}o e Engenharia de Computação}

% local da realização do trabalho
\location{Porto Alegre}{RS}

% data da entrega do trabalho (mês e ano)
\date{12}{2017}


% Palavras chave
\keyword{Palavra-chave1}
\keyword{Palavra-chave2}
\keyword{Palavra-chave3}


%%%%%%%%%%%%%%%%%%%%%%%%%%%%%%%%%%%%%%%%%%%%%%%%%%%%%%%%%%%%%%%%%%%%%%%%%%%%%%%%%%%%%%
% Início do documento e elementos pré-textuais
%%%%%%%%%%%%%%%%%%%%%%%%%%%%%%%%%%%%%%%%%%%%%%%%%%%%%%%%%%%%%%%%%%%%%%%%%%%%%%%%%%%%%%

% Declara início do documento
\begin{document}

% inclui folha de rosto
\maketitle

\selectlanguage{brazilian}

% Sumario
\tableofcontents



%%%%%%%%%%%%%%%%%%%%%%%%%%%%%%%%%%%%%%%%%%%%%%%%%%%%%%%%%%%%%%%%%%%%%%%%%%%%%%%%%%%%%
% Aqui comeca o texto propriamente dito
%%%%%%%%%%%%%%%%%%%%%%%%%%%%%%%%%%%%%%%%%%%%%%%%%%%%%%%%%%%%%%%%%%%%%%%%%%%%%%%%%%%%%

%espaçamento entre parágrafos
%\setlength{\parskip}{6 pt}

\selectlanguage{brazilian}



%%%%%%%%%%%%%%%%%%%%%%%%%%%%%%%%%%%%%%%%%%%%%%%%%%%%%%%%%%%%%%%%%%%%%%%%%%%%%%%%%%%%%
% Intro
%

\chapter{INTRODUÇÃO} \label{intro}

Escolhemos para implementar o problema do jogo Tower Defense.

Tower Defense é um subgênero de jogos de estratégia, onde o jogador deve defender seu território (que varia de acordo com a temática do jogo) contra o ataque de inimigos, podendo utilizar um montante inicial para adquirir elementos do jogo que o ajudem na defesa.

Nosso código-fonte (além de zipado junto com este relatório) encontra-se no seguinte repositório:
\texttt{\\\tab https://github.com/gutoboranga/tower-defense}

%%%%%%%%%%%%%%%%%%%%%%%%%%%%%%%%%%%%%%%%%%%%%%%%%%%%%%%%%%%%%%%%%%%%%%%%%%%%%%%%%%%%%

\chapter{PARADIGMA} \label{intro}

Para a primeira etapa do trabalho, optamos por implementar o jogo usando Orientação a Objetos devido a conhecimento prévio de ambos integrantes do grupo.

Acreditamos que será mais fácil implementar o jogo no paradigma funcional como segunda etapa ao invés da primeira, pois agora temos um conhecimento maior acerca da linguagem e das regras do próprio jogo.

%%%%%%%%%%%%%%%%%%%%%%%%%%%%%%%%%%%%%%%%%%%%%%%%%%%%%%%%%%%%%%%%%%%%%%%%%%%%%%%%%%%%%

\chapter{REQUISITOS} \label{intro}

A seguir está a lista de requisitos para o paradigma de Orientação a Objetos.

Para cada requisito cumprido na implementação, há um exemplo retirado do nosso código-fonte.

%%%%%%%%%%%%%%%%%%%%%

\section{Classes}

Especificar e utilizar classes (utilitárias ou para representar as estruturas de dados utilizadas pelo programa).

Utilizamos diversas classes no projeto inteiro, seguindo o padrão MVC com leves alterações devido à presença do conceito de GameScene, trazido pela biblioteca de jogos SpriteKit.

Exemplo: classe \texttt{Projectile}, que representa um projétil atirado pelas torres do jogo.
\texttt{\\class Projectile: StandardBlock \{\\\tab ...\\\}}

%%%%%%%%%%%%%%%%%%%%%

\section{Encapsulamento}

Fazer uso de encapsulamento e proteção dos atributos, com os devidos métodos de manipulação (setters/getters) ou propriedades de acesso, em especial com validação dos valores (parâmetros) para que estejam dentro do esperado ou gerem exceções caso contrário.

Exemplo:

Praticamos onde possível (e necessária) a proteção dos atributos. Por exemplo, na classe \texttt{LifeBar}, que representa a porcentagem de vida dos personagens do jogo (e seu elemento visual, no formato de uma barra que muda de acordo com o valor associado), há as seguintes funções de get e set.

\texttt{\\public func getLife() -> Double \{\\\tab return self.actualLife\\\}}

\texttt{\\public func setLife(newValue: Double) \{\\\tab self.actualLife = newValue\\\}}
    
O objetivo aqui é proteger o atributo privado \texttt{actualLife}, que representa o quanto de vida o objeto que possui este \texttt{LifeBar} ainda possui dentro do jogo.

%%%%%%%%%%%%%%%%%%%%%

\section{Construtores-Padrão}

Especificação e uso de construtores-padrão para a inicialização dos atributos e, sempre que possível, de construtores alternativos.

Na maioria das classes implementadas, criamos apenas um construtor - com parâmetros - pois não encontramos necessidade de criar construtores padrão, uma vez que em Swift pode-se inicializar os atributos de uma classe junto com sua declaração. Ao adotar esta prática, os atributos já terão valores default na criação do objeto.

Na classe \texttt{Enemy}, temos um construtor padrão que inicializa os elementos que ainda não haviam sido inicializados na declaração e chamamos o método \texttt{init} da classe pai (super.init(parâmetros)).

%%%%%%%%%%%%%%%%%%%%%

\section{Destrutores}

Especificação e uso de destrutores (ou métodos de finalização), quando necessário.

Não houve necessidade de criar destrutores nas classes implementadas pois o framework utilizado lida com a desalocação de memória de maneira competente.

%%%%%%%%%%%%%%%%%%%%%

\section{Espaços de nome}

Organizar o código em espaços de nome diferenciados, conforme a função ou estrutura de cada classe ou módulo de programa.

Não foi implementado.

%%%%%%%%%%%%%%%%%%%%%

\section{Herança}

Usar mecanismo de herança, em especial com a especificação de pelo menos três níveis de hierarquia, sendo pelo menos um deles correspondente a uma classe abstrata, mais genérica, a ser implementada nas classes-filhas.

Exemplo:

Temos a classe \texttt{StandardBlock}, que é uma classe abstrata da  qual a maioria dos personagens do jogo herda. \texttt{StandardBlock} possui apenas um atributo de orientação no espaço e métodos para manipular este atributo.

Dentre suas herdeiras está a classe \texttt{Enemy}, que representa o tipo genérico de todos os inimigos presentes no jogo. Esta classe possui atributos e métodos comuns a todas as especificações de inimigos.

Por sua vez, \texttt{Enemy} possui algumas classes herdeiras que especializam e personalizam o conceito genérico de inimigo, com implementações específicas de funções genéricas de acordo com as características deste personagem. Por exemplo, temos a classe \texttt{AstronautEnemy}, que representa o personagem astronauta.

%%%%%%%%%%%%%%%%%%%%%

\section{Polimorfismo por inclusão}

Utilizar polimorfismo por inclusão (variável ou coleção genérica manipulando entidades de classes filhas, chamando métodos ou funções específicas correspondentes).

Como citado anteriormente na seção Herança, temos a classe \texttt{Enemy} representando a classe genérica dos inimigos no jogo. Ela possui alguns métodos que devem ser sobrecarregados pelas suas herdeiras, de acordo com as características destas.

Por exemplo, o método \texttt{getDamageValue()}, que retorna um Double representando o valor do dano que este inimigo causa ao atingir o castelo do jogador. Para a classe \texttt{AstronautEnemy} (que no âmbito do jogo representa um ser humano), esta função é implementada de forma a retornar o valor 0.7, enquanto que na classe \texttt{RoverEnemy} (que representa um veículo de exploração espacial, portanto mais forte do que um ser humano), a implementação desta função retorna o valor 2.0.

%%%%%%%%%%%%%%%%%%%%%

\section{Polimorfismo paramétrico}

Usar polimorfismo paramétrico através da especificação de \textit{algoritmo} (método ou função genérico) utilizando o recurso oferecido pela linguagem (i.e., generics, templates ou similar) e da especificação de \textit{estrutura de dados} genérica utilizando o recurso oferecido pela linguagem.

Exemplo:

Implementamos uma função que previne erros na física da colisão dos elementos do jogo. Esta função recebe dois parâmetros de qualquer tipo e retorna um booleano que indica se seu tipo (mais genérico) é o mesmo.

\texttt{\\func sameType<T,E>(one : T, two: E) -> Bool \{\\\tab if type(of: one) == type(of: two) \{\\\tab \tab return true\\\tab\}\\\tab return false\\\}}

%func sameType<T,E>(one : T, two: E) -> Bool {
%    if type(of: one) == type(of: two) {
%        return true
%    }
%    return false
%}

%%%%%%%%%%%%%%%%%%%%%

\section{Polimorfismo por sobrecarga}

Usar polimorfismo por sobrecarga (vale construtores alternativos).

Utilizamos polimorfismo por sobrecarga na classe \texttt{Enemy}, onde temos duas implementações (com assinaturas diferentes) para a função \texttt{move()}:

\begin{itemize}
\setlength{\itemindent}{1em}
    \item Uma delas é pública e não recebe parâmetros, tendo a seguinte assinatura, portanto:
    
    \texttt{public func move()}
    
    Esta serve para iniciar o movimento de um inimigo, e é chamada de fora da classe que a implementa, indicando o início do movimento de um inimigo.
    
    \item A segunda é privada e recebe como parâmetro uma lista de movimentos (pares ordenados x, y) a serem feitos pelo inimigo em questão. Sua assinatura é a seguinte:
    
    \texttt{private func move(\_ newDir : [[CGFloat]])}
    
    Nesta função privada, é realizado o movimento imeira posição (x, y) da lista recebida e após, são feitas chamadas recursivas para ela mesma enviando a lista sem o primeiro elemento.
\end{itemize}


%- A segunda é privada e recebe como parâmetro uma lista de movimentos pares ordenados (x, y). Sua assinatura é a seguinte::
%private func move(_ newDir : [[CGFloat]])
%Nesta função privada, é realizado o movimento para a primeira posição (x, y) da lista recebida e após, são feitas chamadas recursivas %para ela mesma enviando a lista sem o primeiro elemento.

%%%%%%%%%%%%%%%%%%%%%

\section{Delegates}

Especificar e usar delegates.

Utilizamos o conceito de delegate (em Swift conhecido como protocol) para fazer a comunicação entre os elementos visuais do jogo (View, no padrão MVC) com a parte mais conceitual (Model) sem criar dependências desnecessárias entre estas classes.

Exemplo:

Na classe \texttt{Tower} (que modela o conceito de uma torre de defesa dentro do jogo), é declarado o seguinte delegate:

\texttt{\\ protocol TowerDelegate \{\\\tab func removeTower(tower : Tower)\\\tab func upgradeTower(tower: Tower)\\\}}

A classe que implementa este delegate é a \texttt{GameScene}. Ela é, portanto, a responsável por tomar as ações necessárias quando a classe \\texttt{Tower} acionar algum dos métodos declarados neste protocol. Estas ações envolvem chamadas a métodos de outras classes. O uso deste delegate evita que a classe \texttt{Tower} fique acoplada a estas outras classes, deixando que a \texttt{GameScene} faça a comunicação necessária entre elas.


%%%%%%%%%%%%%%%%%%%%%%%%%%%%%%%%%%%%%%%%%%%%%%%%%%%%%%%%%%%%%%%%%%%%%%%%%%%%%%%%%%%%%
% Conclusão
%
\chapter{CONCLUSÃO}

Ao término deste etapa do trabalho, podemos concluir que utilizar Orientação a Objetos em projetos que envolvem um grande número de entidades e elementos com diferentes características é muito apropriado.

Isto se deve ao fato de este paradigma se basear na criação de tipos abstratos (com seus atributos e métodos próprios), que permite uma maior separação nas responsabilidades de cada módulo do programa, resultando em um código-fonte mais organizado e conciso.


%%%%%%%%%%%%%%%%%%%%%%%%%%%%%%%%%%%%%%%%%%%%%%%%%%%%%%%%%%%%%%%%%%%%%%%%%%%%%%%%%%%%%
% Referências
%
\chapter{REFERÊNCIAS}

Acessos feitos em 28/10:

\begin{itemize}
\setlength{\itemindent}{1em}
    \item https://developer.apple.com/swift/
    \item https://insights.stackoverflow.com/survey/2017
    \item http://www.bestprogramminglanguagefor.me/why-learn-swift
    \item https://pt.stackoverflow.com/questions/141624/o-que-%C3%A9-paradigma
    \item https://swift.org
    \item http://gameranx.com/features/id/13529/article/best-tower-defense-games/
\end{itemize}

\end{document}
