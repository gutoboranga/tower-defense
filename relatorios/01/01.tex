%%%%%%%%%%%%%%%%%%%%%%%%%%%%%%%%%%%%%%%%%%%%%%%%%%%%%%%%%%%%%%%%%%%%%%%%%%%%%%%%%%%%%%
% Modelo de relatório de Disciplina de MLP a partir da
% classe latex iiufrgs disponivel em http://github.com/schnorr/iiufrgs
%%%%%%%%%%%%%%%%%%%%%%%%%%%%%%%%%%%%%%%%%%%%%%%%%%%%%%%%%%%%%%%%%%%%%%%%%%%%%%%%%%%%%%

%%%%%%%%%%%%%%%%%%%%%%%%%%%%%%%%%%%%%%%%%%%%%%%%%%%%%%%%%%%%%%%%%%%%%%%%%%%%%%%%%%%%%%
% Definição do tipo / classe de documento e estilo usado
%%%%%%%%%%%%%%%%%%%%%%%%%%%%%%%%%%%%%%%%%%%%%%%%%%%%%%%%%%%%%%%%%%%%%%%%%%%%%%%%%%%%%%
%
\documentclass[rel_mlp]{iiufrgs}

%%%%%%%%%%%%%%%%%%%%%%%%%%%%%%%%%%%%%%%%%%%%%%%%%%%%%%%%%%%%%%%%%%%%%%%%%%%%%%%%%%%%%%
% Importação de pacotes
%%%%%%%%%%%%%%%%%%%%%%%%%%%%%%%%%%%%%%%%%%%%%%%%%%%%%%%%%%%%%%%%%%%%%%%%%%%%%%%%%%%%%%
% (a A seguir podem ser importados os pacotes necessários para o documento, de acordo
% com a necessidade)
%
\usepackage[brazilian]{babel}	    % para texto escrito em pt-br
\usepackage[utf8]{inputenc}         % pacote para acentuação
\usepackage{graphicx}         	    % pacote para importar figuras
\usepackage[T1]{fontenc}            % pacote para conj. de caracteres correto
\usepackage{times}                  % pacote para usar fonte Adobe Times
\usepackage{enumerate}              % para lista de itens com letras
\usepackage{breakcites}
\usepackage{titlesec}
\usepackage{enumitem}
\usepackage{titletoc}
\usepackage{listings}			    % para listagens de código-fonte
\usepackage{mathptmx}               % p/ usar fonte Adobe Times nas formulas matematicas
\usepackage{url}                    % para formatar URLs
%\usepackage{color}				    % para imagens e outras coisas coloridas
%\usepackage{fixltx2e}              % para subscript
%\usepackage{amsmath}               % para \epsilon e matemática
%\usepackage{amsfonts}
%\usepackage{setspace}			    % para mudar espaçamento dos parágrafos
%\usepackage[table,xcdraw]{xcolor}  % para tabelas coloridas
%\usepackage{longtable}             % para tabelas compridas (mais de uma página)
%\usepackage{float}
%\usepackage{booktabs}
%\usepackage{tabularx}
%\usepackage[breaklinks]{hyperref}

\usepackage[alf,abnt-emphasize=bf]{abntex2cite}	% pacote para usar citações abnt

% isso meio que coloca helvetica no título dos capítulos, mas fica com muito espaçamento
%\titleformat{\chapter}
%  {\normalfont\fontsize{12}{0}\sffamily\bfseries}
%  {\thechapter}
%  {1em}
%  {}

  
% isso muda a fonte de tudo:
% \renewcommand\familydefault{\sfdefault}

%%%%%%%%%%%%%%%%%%%%%%%%%%%%%%%%%%%%%%%%%%%%%%%%%%%%%%%%%%%%%%%%%%%%%%%%%%%%%%%%%%%%%%
% Macros, ajustes e definições
%%%%%%%%%%%%%%%%%%%%%%%%%%%%%%%%%%%%%%%%%%%%%%%%%%%%%%%%%%%%%%%%%%%%%%%%%%%%%%%%%%%%%%
%

% define estilo de parágrafo para citação longa direta:
\newenvironment{citacao}{
    %\singlespacing
    %\footnotesize
    \small
    \begin{list}{}{
        \setlength{\leftmargin}{4.0cm}
        \setstretch{1}
        \setlength{\topsep}{1.2cm}
        \setlength{\listparindent}{\parindent}
    }
    \item[]}{\end{list}
}

% adiciona a fonte em figuras e tabelas
\newcommand{\fonte}[1]{\\Fonte: {#1}}

% Ative o seguinte caso alguma nota de rodapé fique muito longa e quebre entre múltiplas
% páginas
%\interfootnotelinepenalty=10000

%%%%%%%%%%%%%%%%%%%%%%%%%%%%%%%%%%%%%%%%%%%%%%%%%%%%%%%%%%%%%%%%%%%%%%%%%%%%%%%%%%%%%%
% Informações gerais
%%%%%%%%%%%%%%%%%%%%%%%%%%%%%%%%%%%%%%%%%%%%%%%%%%%%%%%%%%%%%%%%%%%%%%%%%%%%%%%%%%%%%%

% título
\title{Tower Defense e Swift: Uma Dupla da Pesada}

% autor
%\author{Autor2}{Aluno} % {sobrenome}{nome} 1 para cada aluno
\author{Boranga}{Augusto} % {sobrenome}{nome}
\author{Franzoi Scroferneker}{Rodrigo} % {sobrenome}{nome}

% Professor orientador da disciplina
\advisor[Prof.~Dr.]{Mello Schnorr}{Lucas}

% Nome do(s) curso(s):
\course{Curso de Graduação em Ciência da Computa{\c{c}}{\~a}o e Engenharia de Computação}

% local da realização do trabalho
\location{Porto Alegre}{RS}

% data da entrega do trabalho (mês e ano)
\date{12}{2017}


% Palavras chave
\keyword{Palavra-chave1}
\keyword{Palavra-chave2}
\keyword{Palavra-chave3}


%%%%%%%%%%%%%%%%%%%%%%%%%%%%%%%%%%%%%%%%%%%%%%%%%%%%%%%%%%%%%%%%%%%%%%%%%%%%%%%%%%%%%%
% Início do documento e elementos pré-textuais
%%%%%%%%%%%%%%%%%%%%%%%%%%%%%%%%%%%%%%%%%%%%%%%%%%%%%%%%%%%%%%%%%%%%%%%%%%%%%%%%%%%%%%

% Declara início do documento
\begin{document}

% inclui folha de rosto
\maketitle

\selectlanguage{brazilian}

% Sumario
\tableofcontents



%%%%%%%%%%%%%%%%%%%%%%%%%%%%%%%%%%%%%%%%%%%%%%%%%%%%%%%%%%%%%%%%%%%%%%%%%%%%%%%%%%%%%
% Aqui comeca o texto propriamente dito
%%%%%%%%%%%%%%%%%%%%%%%%%%%%%%%%%%%%%%%%%%%%%%%%%%%%%%%%%%%%%%%%%%%%%%%%%%%%%%%%%%%%%

%espaçamento entre parágrafos
%\setlength{\parskip}{6 pt}

\selectlanguage{brazilian}



%%%%%%%%%%%%%%%%%%%%%%%%%%%%%%%%%%%%%%%%%%%%%%%%%%%%%%%%%%%%%%%%%%%%%%%%%%%%%%%%%%%%%
% Problema
%

\chapter{PROBLEMA} \label{intro}

Neste capítulo daremos um panorama a respeito do problema escolhido.


%%%%%%%%%% Seção 1

\section{Sobre o objetivo}

O objetivo é defender as torres.


\section{Historicamente}

Não sei ainda, vamos pesquisar.


\section{Exemplos famosos}

Aqui haverá alguns bons exemplos.



%%%%%%%%%%%%%%%%%%%%%%%%%%%%%%%%%%%%%%%%%%%%%%%%%%%%%%%%%%%%%%%%%%%%%%%%%%%%%%%%%%%%%
% Linguagem
%

\chapter{LINGUAGEM} \label{intro}

Neste capítulo daremos um panorama a respeito da linguagem adotada, explanando acerca de suas origens e discutindo seus pontos positivos e negativos.

\section{Sobre a origem}

Swift foi criado pelo Jobs.

\section{Sobre a utilização}

Swift é pra fazer app.

Também pra servidores.

E só, acho.

\section{Sobre o motivo da escolha}

Gostamos.

\section{Análise}

A análise crítica será feita ao término do trabalho.


%%%%%%%%%%%%%%%%%%%%%%%%%%%%%%%%%%%%%%%%%%%%%%%%%%%%%%%%%%%%%%%%%%%%%%%%%%%%%%%%%%%%%
% Capítulo 2
%
\chapter{IMPLEMENTAÇÃO}

Esta seção será construída no futuro.


%%%%%%%%%%%%%%%%%%%%%%%%%%%%%%%%%%%%%%%%%%%%%%%%%%%%%%%%%%%%%%%%%%%%%%%%%%%%%%%%%%%%%
% Conclusão
%
\chapter{CONCLUSÃO}

Esta seção será construída no futuro.


%%%%%%%%%%%%%%%%%%%%%%%%%%%%%%%%%%%%%%%%%%%%%%%%%%%%%%%%%%%%%%%%%%%%%%%%%%%%%%%%%%%%%
% Referências
%
\chapter{REFERÊNCIAS}

Esta seção será construída no futuro.

\end{document}
